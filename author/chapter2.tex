\chapter{Getting Started}
\section{Insertion Sort}
\textbf{Illustration:} 
\\
\textbf{Pseudocode:}  
\begin{lstlisting}[language=Python, numbers=left]
def InsertionSort(arr[]):
	for j=1 to arr.length - 1:
		value = arr[j]
		i = j - 1
		while i >= 0 and arr[i] > value:
			arr[i + 1] = arr[i]
			i = i - 1
		arr[i + 1] = value
	return arr
\end{lstlisting}
\textbf{Explanation:}
\begin{enumerate}
	\item Declare function with one parameter the input array arr.
	\item Iterate index $j$ from 0 to the last index (one less than length of the array).
	\item Put value of element $arr[j]$ in the variable $value$ 
	\item For each iteration, get the previous index $i=j-1$.
	\item If $i\geq0$ which means $arr[i]$ is not out of our array, and $arr[i]$ is bigger than $value$
	\item Move $arr[i]$ to the right by setting $arr[i+1]=i$
	\item Once we iterated through all the element before $value$ or we reached a value smaller than $value$, put $value$ at $arr[i+1]$ where it should be in the sorted sub array.
	\item Return the array $arr$ that is now sorted
\end{enumerate} 

\section{Loop Invariant}
How do we know our algorithm is correct? One way to check is to use the loop Invariant. There are 
three part of loop invariant. \\\\
\textbf{Initialization:} Loop invariant is true before the loop begins.\\
\textbf{Maintenance:} It will remain true after every single iteration.\\
\textbf{Termination:} The loop invariant is still true, but it also has to show that the algorithm produces correct output.\\\\
\vspace*{0.2cm}{\large\textbf{Loop Invariant of Insertion Sort}}\\ 
Let's look at the loop invariant of insertion sort as an example.\\

\section{Analyzing Algorithms}
\section{Designing Algorithms with Divide and Conquer}